%#! platex thesisRev; dvipdfmx thesisRev

\documentclass{jreport}
\usepackage{thesis}

% \usepackage{enumerate}
% \usepackage{amsmath}
% \usepackage{url}
% \usepackage{comment}
% \usepackage{booktabs}

% \usepackage[dvipdfmx]{graphicx}
% \usepackage{fancyvrb}
% \fvset{frame=single,numbers=left,numbersep=3pt}

\newcommand\SMLSharp{$\mbox{SML\fontfamily{ptm}\selectfont\#}$}

\title{タイトル}

\author{著者}{ID9999}
\eauthor{The Author}
\school{東北大学工学部}
\adviser{大堀 淳\ 教授,菊池 健太郎\ 助教}

\date{令和4年1月31日}{令和4年1月}

\renewcommand{\bibname}{参考文献}

\begin{document}
\maketitle


\chapter*{概要}
概要

\setcounter{tocdepth}{1}
\tableofcontents

\chapter{序論}
\section{背景と目的}
背景と目的

\section{関連研究}
関連研究

参考文献は,
\cite{bunken1}
と
\cite{bunken2}
さえに,
\cite{robobug:2017}
である.



\section{本論文の構成}
本論文の構成

\chapter{題2章}
題2章
\section{題2章題1節}
題2章題1節

\chapter{デバッグ技術}
\chapter{デバッグ学習環境システムの設計と実装}
\chapter{関連研究と議論}

\chapter{結論}
結論

\chapter*{謝辞}
{謝辞}

\appendix
\chapter{Appendix}
Appendix
\section{sec1}
sec1


% \bibliographystyle{junsrt}
% \bibliography{thesisrefs}

\begin{thebibliography}{10}

\bibitem{bunken1}
著者1
\newblock 参考文献1

\bibitem{bunken2}
著者2
\newblock 参考文献2

\bibitem{robobug:2017}
Jeremy S.~Bradbury Michael A.~Miljanovic.
\newblock Robobug: A serious game for learning debugging techniques.
\newblock In {\em the 2017 ACM Conference}, 2017.

\end{thebibliography}

\end{document}

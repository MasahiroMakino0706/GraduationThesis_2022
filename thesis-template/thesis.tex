\documentclass{jreport}
\usepackage{thesis}
\usepackage{enumerate}
\usepackage{amsmath}
\usepackage{url}
\usepackage[dvipdfmx]{graphicx}
%\usepackage{comment}
\usepackage{booktabs}
\usepackage{fancyvrb}

\newcommand\SMLSharp{$\mbox{SML\fontfamily{ptm}\selectfont\#}$}
\fvset{frame=single,numbers=left,numbersep=3pt}

\title{ゲーミフィケーションを利用したデバッグ技術学習法の\\考案と学習支援システムの実装}

\author{川原田伸}{B7TB2507}
\eauthor{Shin Kawarada}
\school{東北大学工学部}
\adviser{大堀 淳\ 教授,上野 雄大\ 准教授,菊池 健太郎\ 助教}

\date{令和3年3月8日}{令和3年3月}

\begin{document}
\maketitle


\chapter*{概要}
本研究は,ゲームデザイン技法やノウハウを現実世界の活動に応用するゲーミフィケーションを
用いて,プログラミング初学者を対象とした,プログラミングにおけるデバッグ学習の促進を目的とする.
本論文ではまず,ゲーミフィケーションとデバッグ技術についての基礎を述べる.さらに,
目的達成のための第一歩として\SMLSharp により試作した対話型デバッグ学習システムについて述べる.
また,今後の発展と実用化に関する考察を行う.

\setcounter{tocdepth}{1}
\tableofcontents

\chapter{序論}
\section{背景と目的}
あらゆるシステムがプログラムによって動作している現代において,
プログラミング教育の重要性は高まりつつある.
大学や専門学校等でのプログラミング教育に加え,
IT企業への転職を目的としたプログラミングスクールや小学校でのプログラミング教育必修化もあり,
情報工学を専門としない学生や社会人もプログラミングを学ぶ機会が増えている.
多くのプログラミング初学者は,
コンパイルエラーや実行時のエラーに直面した際,
直感的なステートメントの変更や変数の値の出力を行うが,
これは本来期待されるデバッグの方法とは異なり,
学習効率の低下や直感的なデバッグの習慣化が懸念される.
さらに,より実践的で大規模な開発でのデバッグを行う際に適切な対処をすることが難しく,
学習におけるプログラミングと実践的なプログラミングの間に障壁が発生してしまう.
したがって,プログラミング教育と並行し,
デバッグの考え方を教授する機会を設けることは学習者にとって有意義なものであると考えられる.

しかし,プログラミング教育現場でデバッグ自体の講義があることは珍しい.
実践的なデバッグの教本としては文献\cite{Andreas:Debugging}等が存在するが,
このような実践的な技術は多くの情報工学及び数学知識が必要となり,
これを知識量が不足しているプログラミング初学者が自力で理解することは現実的でない.
また,アルゴリズムやデータ構造といったプログラミングの学習は現実の問題解決に直結する能力である.
それに対し,デバッグの学習はプログラミング学習の補助的な立場を取り,
学習者がデバッグの学習を行う意義を理解することが困難であることが考えられる.

本研究では,このようなデバッグ学習に関する問題に対し,
ゲーミフィケーション\cite{Inoue:Gamification}の利用を試みる.
ゲーミフィケーションとは,TVゲーム開発において培われてきた人間を惹きつけるための技術やノウハウを,
教育や社会生活に応用することを指す.
例えば,RPGを遊ぶ感覚で運動を行うことができるフィットネスゲーム「リングフィットアドベンチャー」\cite{ring}は,
健康維持活動に対するゲーミフィケーションの例である.
ゲーミフィケーションを用いることで,
プレイヤーはより意欲的に勉強や社会活動に取り組むことができる.
この例のように,ゲーミフィケーションとデバッグ学習を組み合わせることにより,
デバッグ学習をより意欲的なものとし,
学習障壁を下げることが期待できる.
本研究では,ゲーミフィケーションをプログラミングにおけるデバッグ学習に応用し,
プログラミング学習者のデバッグ学習を促進することを目的とする.

また,本研究では,目的達成のための第一歩として次の2つアプローチを試みる.
1つ目は,書籍等の文章を読んで学習を行う従来の方法では達成が難しい対話的な学習環境を,ソフトウェアにより
実現することである.開発言語及び学習対象言語として関数型言語\SMLSharp を使用し,試作品であるDebuQを作成した.
ソフトウェア設計にはゲーミフィケーションの考え方を取り入れた.
2つ目は,デバッグ学習を体系的にまとめ,難易度の低い項目から無理なく学習を行うことができる
ようにデバッグに関する練習問題を作成することである.
作成した問題はDebuQにて実装し,学習者が対話的に取り組める形にする.

\section{関連研究}
デバッグの学習を体系的にまとめ,
プログラミングの授業に活用した例としては文献\cite{Yama}の授業パッケージと学習支援システム
が存在する.
デバッグの学習をシリアスゲームとして実現した例としてはRoboBUG~\cite{robobug}が存在する.
本研究は,これら2つの間の立場を取り,
主な作成対象を学習支援システムとするが,
学習者の学習意欲向上を目的としてゲーミフィケーション要素を取り入れている.

\section{本論文の構成}
本論文の構成は次の通りである.
2章で,ゲーミフィケーションの概念と各種技術を述べ,
プログラミング学習におけるゲーミフィケーションの適用方法を考察する.
3章で,バグが発生するメカニズムと,各種デバッグ技術の基礎について説明する.
4章で,\SMLSharp による学習環境システムの設計と実装方法について述べる.
5章で,関連研究について述べ,今後の発展と実用化に向けた課題を議論する.
最後に,6章で本論文の結論を述べる.

\chapter{プログラミング学習におけるゲーミフィケーション}

ゲーミフィケーションとは,ゲーム開発に用いられるノウハウ
や技術を,教育や社会生活等,ゲーム以外の活動に応用することを
指す\cite{Inoue:Gamification}.2010年前後から,教育やビジネス利用を目的としたゲームの
実践的な応用論が注目を集め,ゲーミフィケーションに関する本が
続々と出版された\cite{TGS}.
本章では,ゲーミフィケー
ションの概要と各ゲーミフィケーション技術について言及し,
さらにプログラミングへの応用例について言及する.

\section{ゲーミフィケーション技術}
TVゲームにおいて,プレイヤーをゲームに惹きつけるためには種々の
工夫を施す必要がある.TVゲームプレイヤーが行うことは,「液晶画面等
のゲーム画面からゲームに関する情報を受け取り,
ゲームに接続したコントローラを用いてゲームシステムと対話を
行い,この繰り返しから得られる様々な知見をもとにゲーム内で
設定された目標を達成すること」である.より端的に言えば,
「画面を見てボタンを押すこと」であり,単調作業である.ゲーム
とは,この単調作業に人間の特性を利用した意味づけを行い,プレ
イヤーに快楽を与えることと換言できる.
この意味づけはゲームに限らず応用できる.ゲームにおけるプレイヤー
を惹きつける技術を,ゲーム以外に応用したものがゲーミフィケーション
である.

本論文では,ゲーミフィケーションに用いられる,ゲーム開発における
ゲームデザイン面でのノウハウや技術をゲーミフィケーション技術
と呼ぶ.ゲーミフィケーションは,ゲーミフィケーション技術
をゲーム以外の事象に適用し,対象となる人物の動機付けを行う
ことを目的とする.

ここでは,文献\cite{Kishimoto,Inoue:Gamification}による説明と著者による経験に基づいて
分類した以下の8つについて説明する.
このうち,「記録の可視化」と「継続ボーナス」以外の用語は文献\cite{Inoue:Gamification}あるいは
文献\cite{Kishimoto}において使用されている.

\begin{itemize}
  \item 目標設定
  \item 成長の可視化
  \item 称賛演出
  \item 非強制的参加
  \item 記録の可視化
  \item 継続ボーナス
  \item アンロック
  \item 即時フィードバック
\end{itemize}

\subsection{目標設定}
目標設定は,十分に実現可能な目標を利用者に
与えることを指す.このゲーミフィケーション技術の目的は,
ユーザに行動を起こさせることである.

近年のよくデザインされたゲームにおいては,最初に低い達成難易度の目標が
設定されることがしばしばある.ソーシャルゲーム等の,ゲームの熟練度
が比較的低い非ゲーマーを対象としたゲームでは,ゲームを開始して
最初にチュートリアルが設けられていたり,最初のステージが基本操
作のみでクリアできるようなものとなっていることが多い.目標設定は
プレイヤーが自分の力で目標を達成することにより,内発的な動機付け
として機能する.

\subsection{成長の可視化}
\label{ssec:cansee}
目標を達成したことを利用者が感じられるような仕組みを指す.
目標達成までの距離を定量的に表現することで,プレイヤーの動機づけを行う.

RPGなどのゲームにおける,レベルの概念が成長の可視化に相当する.ゲームを
プレイするユーザーは常に自分のレベルを確認できるため,目標まで
あとどれくらいの時間と労力が必要かが可視化されている.しかし,現実世界
においては自分のレベルは単調増加するわけではない.
そのため,ゲーミフィケーションにおいては,ゲームにおけるレベルの概念をそのまま利用できるとは
限らず,工夫して設計する必要がある.

\subsection{称賛演出}
目標達成や,それを目指した頑張りを称賛することを指す.このゲーミ
フィケーション技術の目的は,利用者に自分の成功を認識させ,成長実感
を促して活動の継続や動機づけを行わせることである.

ゲームでは敵を倒したときやレベルアップをしたときに派手な効果音
と演出が再生される.ゲームのプレイヤーはこの演出によりゲームが進ん
でいることを実感し,動機づけを行うことができる.現実の仕事や勉強
では,自分が達成したことを実感する機会はゲームに比べて少ない.
適切な称賛演出により,プレイヤーの動機づけを促進することができる.

\subsection{非強制的参加}
利用者が自ら能動的に活動に参加できるような環境を指す.利用者が
活動したいときに活動することができ,活動したくないときには
活動しなくてもよい,または活動をやめてもよい状態とする.

ゲームは好きなときに開始することができ,止めたいときにやめることが
できる.また,その要素がゲームを楽しむ姿勢につながっている.
文献\cite{Inoue:Gamification}では,次のように述べられている.

\begin{quote}
  強制的にゲームをプレイさせることで、おもしろいゲームをとたんにつ
  まらなくすることもできる。たとえば、学校の授業中、先生がクラスの
  生徒全員に「あと七〇分以内に、マリオを六面までクリアできなかった
  者は単位ナシ!」と宣言すれば、おそらく、『 マリオ』をたのしく遊
  ぶことが難しくなるだろう。あまりに強制的に他人から押しつけられた
  課題は、ゲームをプレイしているという感覚を失わせることになる。
  \cite{Inoue:Gamification}
\end{quote}

\subsection{記録の可視化}
利用者の成果をいつでも参照可能な形にすることを指す.このゲーミフィケ
ーション技術の目的は,プレイヤーが目標達成や成功,自分の頑張りを定量
的に認識し,モチベーションを高めることである.
このゲーミフィケーション技術は成長の可視化と密接に関係し,
多くの場合は記録を可視化することによって成長も可視化される.

レベル制が導入されている大抵のRPGにおいては,自身の
レベルをいつでも参照することができる.もし,自身のレベルを参照できない
ようなシステムであったら,ゲームのプレイヤーは自分が目標まであとどれ
くらい頑張ればよいかがわからず,モチベーションを損ねてしまう.

文献\cite{Inoue:Gamification}冒頭で述べられている,井上明人氏
が作成した\#denkimeterという節電ゲームは,家庭の電気メーターの数値
を記録し,電気メーターの数値の増加を低くすることを目指すゲームである.
\#denkimeterのプレイヤーは,家の中の様々な家電の電源を落とすなどし
て,電力消費量を減らし,戦闘力を高めることを目指す.公式ホームページ\cite{denkimeter}には,
\#denkimeterを利用して知人と対戦を行うためのルールなども示されて
おり,節電をゲームにした例と言える.このゲームのように,実質的には利用者
が自分の成果を確認できるようになっただけでゲームとなる場合もある.

ゲーミフィケーションの例と言えるかは難しいが,スマホ用アプリケーション
「あすけん」\cite{Asuken}
もゲーミフィケーション技術「記録の可視化」を利用して利用者の
モチベーションを維持させている例であると言える.このアプリケーションは食事を
写真にとることで,その食事の栄養などをAIが分析し,記録するアプリである.
食事内容や現在の体重,自分で記録した運動量といった情報から,次の食事内容
や運動量などの適切なアドバイスをもらうことができる.このアプリは,利用者
の消費カロリーと摂取カロリーを可視化することでダイエットの成功率を高めて
いる.このアプリの役割はあくまで記録と提案のみであるにも関わらず,ダイエット
において高い評価を受けていることからも,記録の可視化の重要性が伺える.

\subsection{継続ボーナス}
活動の成果ではなく,活動を行ったこと自体を称賛することを指す.
このゲーミフィケーション技術の目的は,利用者が活動を継続させること
である.

多くのソーシャルゲームでは,アプリケーションを起動することにより,通称ログインボー
ナスと呼ばれるアイテムやゲーム内通貨を受け取ることができる.このシステム
は,ゲーム内の広告を利用者が見る機会を増やすためや,ゲームへの依存度を
高めるためなど,採用理由はそのゲームによりまちまちであるが,結果として
利用者は高い確率でゲームを継続して起動するようになる.

フィットネスゲームであるWii Fitでは,ゲームを起動してからだ測定
と呼ばれる簡単な測定とバランステストを行うと,ゲーム内のカレンダーに
ハンコを押すことができる.トレーニングは行っていないが,Wii Fitを
起動する機会が増えることで,自然にトレーニングを行う回数も増える.

\subsection{アンロック}
ゲームをプレイする初期段階ではできることが限られているが,ゲームを
プレイし続けることでできることが増えていくようなシステムを指す.
具体的には,プレイヤーが使用可能なアイテムや技が段階的に増えること,
行動範囲の拡大などが挙げられる.
このゲーミフィケーション技術の目的は,初心者の利用者に活動の魅力や楽しさ
などを伝えることで,活動を行う動機を与えることである.

アンロックの典型的な例としては,スーパーマリオ等で使用されているステージシステムがある.
ステージシステムは,ゲームのプレイ単位であるステージに挑戦するために,
「それ以前のステージを全てクリアしなければならない」等の条件を設定するものである.
プレイヤーはステージを次々とアンロックし,ゲームクリアを目指す.

文献\cite{Inoue:Gamification}では,絵心教室DSというゲーム
を例にして,次のように述べられている.

\begin{quote}
  まず、かんたんに線を引く練習。そして絵に色を付ける練習。筆のタッチの
  バリエーションによって質感を出す方法……と一つずつ、高度な内容にステ
  ップアップしていく「アンロック」の仕掛けをフルに活用していると言える
  だろう。\cite{Inoue:Gamification}
\end{quote}

\subsection{即時フィードバック}
ゲームプレイ後に,プレイヤーが結果を即座に確認できることを指す.
このゲーミフィケーション技術は「記録の可視化」の一部とみなすことが
できる.記録の可視化では,自分の活動内容をいつでも確認できることを
指すが,即時フィードバックは活動によって得られる結果を確認できる
ことを指す.

フィードバックの速度が遅かったり,自分の状況を確認できない時間が
長いことは,利用者のモチベーションを維持させるという点においては
望ましくない.人の欲望は,時間が経つにつれ弱くなっていくのである
\cite{Inoue:Gamification}.

\section{プログラミングゲーム}
プログラミングをゲーム化しようという試みは数多く存在する.

プログラミング学習サイトProgate~\cite{Progate}では,演習をクリアする毎に経験値が
与えられ,経験値をためてレベルを上げていくシステムとなっている.
各演習では,演習テーマに関する解説スライドで学習を行った後,コーディングを行う課題
に取り組む.自分で書いたコードの実行結果はブラウザ上で即座に確認できる.
プログラミング言語の他にバージョン管理システム
の使用法など,ソフトウェア開発に必要な種々の演習が用意されている.
Progateにおいては,学習の最初から最終演習に取り組む,というようなことはできない.学習者は,簡単な演習
をこなしていき,段階的に学習を進めていく.これはゲーミフィケーション技術における
目標設定が適用されていると見なすことができる.
自分が書いたプログラムとその実行結果が1つの画面上で即座に確認できるシステムは,
ゲーミフィケーション技術における即時フィードバックである.
このように,種々のゲーミフィケーション技術を適用することにより,一層意欲的な学習体験
が得られる.

プログラミングをテーマとしたエンターテイメントとしては,AtCoder~\cite{AtCoder}などの
競技プログラミングがある.AtCoderでは,定期的にプログラミングコンテストが開催
されている.プログラミングコンテストで出題される問題に早く
正確に回答することで得点が得られる.コンテストには通常複数の問題が用意され
ており,各問題の難易度に応じて得られる得点が設定されている.さらに,オンライン
ゲームでのレーティングにあたるシステムが導入されており,自分の実力が数値によって
可視化されている.また,個人でプログラミングの学習を行う際,作成したプログラムで大量の
テストケースを用いたテストを行うことは手間であり,初学者には難易度が高い.その点,
競技プログラミングはプログラムを提出することにより自動でテストが行われ,合否が
すぐに確認できる.従来のプログラミング学習では,課題をクリアするという概念が曖昧
であったことに対し,競技プログラミングでは課題のクリアが明確に提示される.
これがゲーミフィケーション技術における即時フィードバックと同様の効果をもたらし,
学習者の動機づけを促す.

これらの例は,プログラミングに適切なゲーミフィケーション技術を適用し,ゲーム化
を実現できることを示している.そこで,以下2つのゲーミフィケーション技術をプログラミング学習
に適用することを考える.

1つ目はアンロックである.ゲームに限らず,難易度の低い課題から順に
学習を進めていくことはあらゆる学習の正攻法として知られている.しかし,
現在のプログラミング学習においては学習のレベルデザインが確立されている
とは言い難い.これは,プログラミング能力が数学,アルゴリズム,データ構
造などの複数の能力から構成されることによる.また,それぞれの能力は互いに
干渉しており,中学や高校数学における学習の一方向性がない.例えば,アル
ゴリズムを習得するためにはある程度プログラムが書ける必要があり,プロ
グラムを書く場合もある程度アルゴリズムを理解していなければならない.
より具体的には,C言語でソートアルゴリズムを理解するためには,プログラミング知識
として配列の理解が必要である.また,配列の存在意義や,配列を使用するメリットの
理解を確かなものにするためには,種々のアルゴリズムを配列により実装する
ことが不可欠である.

2つ目は即時フィードバックである.自分が書いたプログラムが正しいかどうか
を確かめるためには,プログラムを実行して何度かテストを行う必要がある.
しかし,このテストケースを作り,テストを行うという工程は時間がかかる上,
手作業では十分な量のテストを行えない場合が多い.
これは他の分野の学習と比較しても明らかにフィードバックが遅い.
プログラムで解決したい任意の課題について,プログラムによる課題の実現方法
は一般に無数に存在する.学習段階においてはこの多様性を排除し,正解と不正解を明確に区別することで
即時フィードバックが実現でき,学習意欲向上が期待できる.

このように,プログラミング学習にはゲーム化できる複数の点が存在するため,
プログラミングという分野においてゲーミフィケーションを行うことは十分に可能である
と考えられる.

\section{シリアスゲーム}
教育や軍事演習など,「シリアスな用途」に用いられるゲームをシリアス
ゲームという\cite{Inoue:Gamification}.
シリアスゲームとして
有名なタイトルとして,「フード・フォース」がある.これは,アフリカ
などの食料が不足する地域に食料を運ぶ仕事を疑似体験できるゲームである.
WFP(国連世界食料計画)によってリリースされたゲームであり,このゲーム
を一通り終えると,WFPが従事している仕事の理解を深めることができる.

一方,ゲーミフィケーションといった場合,必ずしもゲームを開発する
必要はない.ゲーム以外の活動にゲームの要素を取り入れれば,それは
ゲーミフィケーションとして成立する.たとえば,男性用小便器に的当て
マークのシールが貼られていることがあるが,これは尿が周囲に飛散する
ことを防ぐ目的を果たしている.文献\cite{Inoue:Gamification}
では,この例はゲーミフィケーションの第一歩として位置づけられている.

ゲーミフィケーションとシリアスゲームの対比として,文献
\cite{Inoue:Gamification}では,次のように述べられている.

\begin{quote}
  「シリアスゲームは社会のさまざまな問題をゲームのなかに持ち込むこと
  だが、ゲーミフィケーションはゲームを社会のさまざまな場所に持ち込む
  こと」\cite{Inoue:Gamification}
\end{quote}

この定義には,事例をゲームと捉えるかどうかという点で多少の曖昧性が含まれる.
例えば,フィットネスゲームであるWii Fitはゲーミフィケーションだろうか.
文献\cite{Inoue:Gamification}では,Wii Fitは健康・身体
分野での,体重を測定することによるゲーミフィケーションに分類されている.
しかし,Wii Fitは,社会の健康問題をゲームの中に持ち込んだものと捉え
ればシリアスゲームである.要するに,Wii Fitがゲームであるのか,
ゲーム以外の活動であるのかによってシリアスゲームかどうかが決まる.

本論文では,文献\cite{Inoue:Gamification}の定義をそのまま用いるのではなく,次のように定義する.
「シリアスな用途で用いることが可能なゲームをシリアスゲームという.そして,
シリアスゲームを教育や社会生活等,ゲーム以外の活動に用いること
はゲーミフィケーションのひとつの形である.」
つまり,シリアスゲームはゲームのひとつであり,シリアスゲームを教育や社会生活等,
ゲーム以外の目的に用いることはゲーミフィケーションであるとする.

\chapter{デバッグ技術}
不正なプログラムの動作や状態,プログラムコードをバグと言い,バグを除去することを
デバッグという.本章ではバグが発生するメカニズムと,デバッグの際に使用される種々の
技術について述べる.以下で使用する用語「欠陥」「感染」「障害」及び
\ref{sec:mechanism}バグが発生するメカニズム,\ref{sec:proc}デバッグの手順,
~\ref{sec:controlFlow}制御フロー,\ref{sec:observation}事実の観察
における記述は文献\cite{Andreas:Debugging}による説明を参照した.

\section{バグが発生するメカニズム}
\label{sec:mechanism}
以降,バグのうち不正なプログラムコードを欠陥,不正なプログラムの動作を障害,
不正なプログラム状態を感染と呼ぶ.欠陥は感染の原因であり,感染がプログラムの
外部から見える形で現れたものが障害である.

すべてのコードはプログラマが書くものである.よって,すべての欠陥
はプログラマによって作られることになる.
ただし,欠陥の存在とプログラマの失敗は同値ではない.不完全な
仕様,モジュール間のインターフェースの非互換など,プログラマだけ
が原因であるとは言えない場合があるためである.時には,誰も間違っていなくても
障害は発生しうる.当初は欠陥でなかったコードも,ソフトウェアの使用
方法の変化により欠陥になることもある.

欠陥により,プログラムの状態はプログラマが意図しないものになる.
これが感染した状態である.
プログラム状態の実態はプログラムに関係する全ての機器におけるレジスタの値である.
したがって,ネットワークに接続された機器におけるプログラムでは,そのネットワークに
接続されている全ての機器の状態がプログラム状態となる.現実にはプログラム
状態にはハードウェア等による境界が設けられる.

プログラムの状態が感染すると,その後のプログラムの状態は多くの
場合現在のプログラムの状態に依存するため,感染はさらなる感染を引き起こす.

感染により,プログラムの振る舞いが,プログラマが想定した振る舞い
とは異なったものとなり,障害として確認される.

\section{デバッグの手順}
\label{sec:proc}
デバッグの手順は,次の7つの段階に分けられる.それぞれの頭文字から,
TRAFFICと呼ばれる.
\begin{enumerate}
  \item Track: 問題を記録する.
  \item Reproduce: 障害を再現する.
  \item Automate: テストケースの自動化と単純化を行う.
  \item Find: コード中の感染源の疑いがある箇所を見つける.
  \item Focus: 感染源の疑いが最も高い箇所に着目する.
  \item Isolate: 感染の連鎖を分離する.
  \item Correct: 欠陥を修正する.
\end{enumerate}

本章では以降,第\ref{ch:implement}章の準備として,Findにおける制御フロー,Focusにおける
事実の観察について説明する.

\section{制御フロー}
\label{sec:controlFlow}
以下に引数nの階乗を計算するC言語の関数fact()を示す.

\begin{Verbatim}
int fact(int n){
    int a = n, b = 1;
    while (a > 0) {
        b = b * a;
        a = a - 1;
    }
    return b;
}
\end{Verbatim}

関数fact()は,関数内で定義された変数aとbに繰り返し書き込みを行うことで
階乗の計算を実現している.この制御フローを図示すると,図\ref{fig:controlFlow}
のようなグラフとなる.
制御フローグラフは情報がどのように波及していくかを示すものである.
ここでは,どのように個々のステートメントが情報フローに影響するか,
どのようにステートメントが情報フローから影響を受けるのかについて
説明する.

\begin{figure}
  \centering
%  \includegraphics[height=8cm]{controlFlow.png}
  \caption{制御フローグラフ}
  \label{fig:controlFlow}
\end{figure}

\subsection{ステートメントが与える影響}
すべてのステートメントは計算に関与しているため,少なくとも潜在的には必ず情報フロー
に影響を与えていることになる.情報フローへの影響は次の2つに分けられる.

\begin{itemize}
  \item 書き込み(Write)\\
  変数に値を代入することを指す.これにより,プログラムの状態を
  変更する事ができる.ここで言うプログラムの状態とは,極めて広い範囲
  を指し,例えばネットワークに接続されたプログラムが実行されればネットワーク
  に接続されているすべてのデバイスの状態が変わる.現実的にはハードウェア
  境界などで考える範囲を限定する.
  \item 制御(Control)\\
  プログラムカウンタを変更することを指す.図\ref{fig:controlFlow}では,
  ステートメント3のwhile文によって,次に実行されるステートメントが4か6に決定
  される.厳密にはプログラムカウンタも変数であるからプログラムの状態の
  一部であるが,ここでは実用的にプログラムカウンタを時間として考える.
\end{itemize}

\subsection{ステートメントが受ける影響}
前述したことは,ステートメントが能動的に情報フローに影響を与える
ということについてであった.一方で,ステートメントは他のステート
メントから影響を受けることがある.以下に例を示す.

\begin{itemize}
  \item 読み込み(Read)\\
  プログラムの状態,つまり変数の値を読み込むことを指す.
  例えば,「v2=v1+1」というステートメント
  は,変数v1から値を読み込んでいるため,変数v1の状態に影響される.書き込み
  (Write)と同様,ここでの「プログラムの状態」は広い範囲を指す.
  \item 実行(Execution)\\
  ステートメントAが影響を受けるためには,ステートメントAが実行されなければ
  ならない.一般に,ステートメントAの実行が,別のステートメントBによって
  潜在的に制御されているならば,ステートメントAはステートメントBに影響を受ける.
\end{itemize}

\subsection{ステートメントの依存関係}
\label{ssec:statementDependencies}
ステートメント間の依存関係は,次の2種類に分類できる.

\subsubsection{データ依存(Data dependence)}
次の条件両方にあてはまる場合,ステートメントBは,ステートメントA
にデータ依存していると言う.
例えば,図\ref{fig:controlFlow}のステートメント5はステートメント1に
データ依存している.
\begin{itemize}
  \item Aがある変数Vに書き込み,その変数VをBが読み込んでいる.
  \item 上記の変数Vについて,制御フローグラフで,AからBへの経路があり,
  それらの経路のうち1つ以上「変数Vが他のステートメントから書き込まれていない」
  経路が存在する.
\end{itemize}

\subsubsection{制御依存(Control dependence)}
ステートメントBの実行が,潜在的にステートメントAに制御されているとき,
BはAに制御依存していると言う.
例えば,図\ref{fig:controlFlow}のステートメント4,5はステートメント3に
制御依存している.

\section{事実の観察}
\label{sec:observation}
実際のプログラムの動作は,観察を行うことにより特定できる.観察テクニックには
古典的なログ出力,対話型デバッガ,事後デバッグ等が存在するが,ここではログ出力
について言及する.

\subsection{観察の目的}
\label{ssec:observation purpose}
前述したように,バグはプログラム状態の感染が原因である.さらに,感染はプログラム中の欠陥
によって発生する.これらのうち,ログ出力ではプログラム状態を観察することで感染が発生して
いるかを判断する.

観察にあたっては,以下の原則を遵守する必要がある.

\begin{itemize}
  \item 本来のプログラム実行に干渉しない. \\
  観察によりプログラム状態が変化してしまうと,本来のプログラム実行でのプログラム状態を
  正しく特定することが困難となる.
  \item 観察対象,観察タイミングを限定する. \\
  プログラム実行はプログラム状態の膨大な連鎖であり,全体を観察することは現実的ではない.
  観察対象とタイミングを限定し,より本質的な観察とする必要がある.
  \item 科学的手法に則る. \\
  闇雲に観察を行うのではなく,仮説を立て,仮説を検証するために観察を行う.
\end{itemize}

\subsection{観察の方法}
事実を観察する最も簡単な方法は,print関数等によるログの出力である.
通常,プログラミング言語にはCにおけるprintf関数,
\SMLSharp{}におけるprint関数等の標準出力関数が用意されている.
しかし,この方法には以下のような難点もある.

\begin{itemize}
  \item ログ出力ステートメントと元のコードとの混在による,コードの煩雑化
  \item 観察対象イベント及び変数の種類が膨大になった際の,出力の煩雑化
  \item print関数の引数を展開することによる,プログラム全体におけるパフォーマンスの低下
\end{itemize}

これらの問題に対応するためには,次のようなことが考えられる.

\begin{itemize}
  \item 出力フォーマットの統一
  \item ログ出力をオン・オフ可能にする
  \item ログ出力の詳細度を変更できるようにする
  \item ログ出力を再利用できるようにしておく
\end{itemize}

\section{デバッグ技術とゲーミフィケーション}
本研究では,\SMLSharp を対象としてデバッグ学習システムを構築する.
ここでは,\SMLSharp におけるデバッグ環境と,
\SMLSharp でのデバッグ体験をどのようにゲーム化するかを考察する.

\subsection{\SMLSharp におけるデバッグ}
\label{sec:smldebug}
これまでに述べたデバッグ技術は,\SMLSharp{}でのプログラミングにも適用することができる.
ただし,Cのような手続き型言語とは異なったアプローチが必要なデバッグ技術もある.
例えば,Cではプリプロセッサにより元のプログラムに干渉せずに事実の観察が行えるが,
\SMLSharp{}にそのような機構は存在しない.
その代わりに,\SMLSharp では対話環境によってデバッグを行うことができる.
対話環境では,次の事が可能である.

\begin{itemize}
  \item ログの出力\\
  \SMLSharp における対話環境では,自分でプリンタ関数を作成することなく,基本型
  と自分で定義した型をプリントすることができる.さらに,プログラム中に現れた変数
  の推論された型も表示される.これにより,プログラマは自分が書いたプログラムの
  型と内容が想定したものと一致しているかを確認できる.
  \item 単体テスト\\
  作成した関数等が正しく振る舞うかをテストすることができる.問題の修正を行う際に,
  その問題の再現が可能な場合は,対話モードで簡単に任意の引数による関数の呼び出し
  を行うことができる.
  \item プログラムに干渉しない\\
  ソースコードに直接プリンタ処理を加える場合,必ず副作用が生じてしまう.
  それに対し,対話環境を使用することでプログラムを外側から観察することが可能であり,
  元のプログラムに干渉する心配がない.
\end{itemize}

\SMLSharp は関数型言語の特性上,変数の値は参照型を除き変化しない.そのため,
誤った値が代入された変数を発見できればそれが感染である.よって,手続き型言語と
比べてデバッグ時に時間的なプログラム状態の側面を考える必要がなく,空間的な誤り
を発見することが目標となる.

\subsection{デバッグのゲーム化}

デバッグ作業では,多くの場合,次に何をすればよいかが不明瞭になる時がある.
これは,効果的なデバッグ技術の使用方法を知らないプログラミング初学者にとっては挫折の原因となりうる.
デバッグをゲーム化するためには,ゲーミフィケーション技術の成長の可視化(\ref{ssec:cansee})で述べたように,
学習者が作業の前進を通じて自分自身の成長を実感できることが重要である.
本研究では,具体的な障害の例とその解決策を学習者が取り組む演習問題として作成することにより,
各種デバッグ技術の存在とその使用方法を教授することを目指す.

\chapter{デバッグ学習環境システムの設計と実装}
\label{ch:implement}
デバッグ作業なしでプログラミングの学習を行うことは困難である.
しかし,学校の授業等でデバッグ自体の方法を学ぶ機会はほとんどない.
多くの学習者は明確な根拠なしにログを出力し直感的にバグを発見するが,
これでは実践的なデバッグ技術が必要になった際に適切なデバッグが行える
とは考えにくい.

そこで本研究では,プログラミング学習者のデバッグ学習を促進するための第一歩として,
学習環境DebuQ(デバック)を\SMLSharp にて構築した.

\section{設計}
ここでは,実装したプログラムの外部仕様を説明する.

開発した学習環境は,Linuxのshell上で動作する対話型コンソールアプリケーションである.
学習の流れはProgate~\cite{Progate}を参考にした.学習者は学習したいタスクを選び,
学習テーマに関する説明を読んだ後に,出題される問題に回答する.なお,本プログラムでは
タスク開始から終了までの一連の流れをまとめてステージと呼ぶことにする.これをまとめると,
学習者は以下の項目を繰り返しこなすことによりデバッグ技術を獲得できるようになっている.

\begin{enumerate}
  \item 表示されるステージ一覧から,ステージを選択する.
  \item 表示される説明を読む.
  \item 表示される問題を解く.
  \item ステージをクリアしたら,1.に戻る.
\end{enumerate}

グラフィックを使用せずにコンソールアプリケーションとした理由は,開発期間が短いこと,
デバッグ学習の内容を考察することを主な目的としたためであるが,shell上で動作することによ
り,多くの環境で応用が可能となる利点も存在する.
DebuQはshell上で動作するため,例えばshell機能を持つ任意のエディタ内のshellで動作させることが可能である.
これにより,学習者はコーディングを行いながら学習を進めることも可能である.

DebuQで用いているゲーミフィケーション技術は,以下の2つである.

\begin{itemize}
  \item 即時フィードバック
  \item アンロック
\end{itemize}

学習者は,問題に回答するたびにフィードバックとして回答の評価が得られる.
正答した場合だけでなく,誤答をした場合にもなぜ間違えたかを説明するため,
学習の躓きを防ぐことが期待できる.

また,DebuQ開始時のステージは1項目のみ表示されており,
ステージをクリアする毎に新しいステージがアンロックされる仕組みとなっている.
これにより,学習者は次に何を学習すれば良いかを具体的に知ることができるため,
モチベーションの維持が期待できる.

DebuQで学習できる内容を以下に示す.

\begin{itemize}
  \item エラー出力の読み取り
  \item 科学的手法
  \item ステートメントの依存関係
  \item ログの出力
\end{itemize}

エラー出力の読み取りは,
文法ミス等によって発生するコンパイルエラーからプログラムの欠陥の場所を特定する能力である.
これは比較的簡単に解決できるものの一つであるが,
プログラミング初学者がこのエラーを見た場合,
母国語以外によるエラーメッセージの表示や例外名などの出力により混乱してしまう可能性がある.
DebuQでは,エラーメッセージのうち注目する箇所を行番号に限定し,
欠陥の修正を試みる課題を採用した.

科学的手法とは,障害に対して既に確認された事実と矛盾しない仮説を立て,
その仮説の真偽を確かめることにより欠陥の場所を絞り込む手法のことである.
DebuQでは,\SMLSharp の対話環境によるログの出力を通じて仮説を検証する問題を用意した.

ステートメントの依存関係は,\ref{ssec:statementDependencies}で述べた依存関係のことである.
このうち,DebuQでは主にデータ依存の関係を追跡する課題を用意した.

ログの出力では,\ref{sec:observation}で述べた方法により,障害に関する事実を観察する.

\section{実装}
\label{sec:implement}
作成したプログラムは,次の2つのモジュールから成る.
メインルーチンのプログラムは付録\ref{app:main.sml}に,
問題演習ルーチンのプログラムは付録\ref{app:stage.sml}に示した.

\begin{itemize}
  \item メインルーチン\\
  ステージの選択,挑戦できるステージの表示,達成状況の表示等,プログラム全体の
  制御を行う.
  \item 問題演習ルーチン\\
  指定された問題ファイルの読み込み,説明の表示と問題の出題,入力の受付,正誤判定等の
  各ステージにおける処理を行う.
\end{itemize}

メインルーチンmain.smlでは,main関数に学習の進行状況を示す変数envを渡し,全てのステージをクリア
するまでループを行う.変数envの型は以下のレコード型である.

\begin{verbatim}
 env : {task : string list, appeare : int list, unlock : int option}
\end{verbatim}

task属性は,各ステージのタスク名をリストで保持する.appeare属性は,プレイヤーが挑戦可能
なステージの番号をリストで保持する.unlock属性はアンロックするステージの番号を表し,
直前に挑戦したステージをクリアしたなら,その次のステージ番号をnとして\verb|SOME n|,
直前にステージをクリアしていなかったら\verb|NONE|とする.

stage.smlでは,問題ファイルの字句解析を行うLexerストラクチャ,構文解析を行うParserストラクチャ,
文章の表示や問題の出題等を行うPlayストラクチャの3つが定義されている.
stage.smlにおける字句解析を行うストラクチャLexerのシグネチャLEXERを以下に示す.

\begin{Verbatim}
signature LEXER =
sig
  exception formatError

  datatype command =
    Select
  | SomeSelect
  | Answer of int * bool
  | Input
  | SomeInput

  datatype token =
    Statement of string
  | Split
  | Begin of command
  | End
  
  val lexer : string -> token
  
  val fileLexer : string -> token list
end
\end{Verbatim}

command型は,コマンドを示す型である.
token型は,トークンを示す型である.
問題ファイル中の1行が1つのトークンに対応し,トークンは文章Statement,文章の区切りSplit,
コマンドの開始Begin,コマンドの終了Endからなる.
Selectは選択問題,SomeSelectは複数選択問題,Answerは選択問題の選択肢,
Inputは文字列を答える問題,SomeInputは複数の文字列を答える問題を表している.
lexer関数は,一行のstring型の値を受け取り,それをtoken型のデータに変換する.
fileLexer関数は,ファイル名をstring型の値として受け取り,対象ファイルの各行を
token型に変換した値のリストを返す.

以下に,サンプル問題ファイルsample.txtを示す.

\begin{Verbatim}
問題です.
(1) 正解
(2) 不正解
(3) 不正解

<begin select>
<begin answer 1 true>
(1)は正解です.
<end>
<begin answer 2 false>
(2)は不正解です.
<end>
<begin answer 3 false>
(3)は不正解です.
<end>
<end>
\end{Verbatim}

sample.txtを字句解析した結果を以下に示す.

\begin{Verbatim}
  # Lexer.fileLexer "sample.txt";
  val it =
    [
      Statement "問題です.\n",
      Statement "(1) 正解\n",
      Statement "(2) 不正解\n",
      Statement "(3) 不正解\n",
      Statement "\n",
      Begin Select,
      Begin (Answer (1, true)),
      Statement "(1)は正解です.\n",
      End,
      Begin (Answer (2, false)),
      Statement "(2)は不正解です.\n",
      End,
      Begin (Answer (3, false)),
      Statement "(3)は不正解です.\n",
      End,
      End
    ]
    : Lexer.token list  
\end{Verbatim}

sample.txtにおける"\verb|<|"と"\verb|>|"で囲まれた行はコマンドを示しており,その他の行は学習者にそのまま
表示する文章として字句解析が行われている.各Answerコマンドで指定された文章は,学習者がその選択肢を
選んだ際に表示される文章を表している.

構文解析処理を行うストラクチャParserのシグネチャPARSERを以下に示す.

\begin{Verbatim}
  signature PARSER =
  sig
    datatype qtype =
      Statement of string
    | Split
    | Select of {index : int, pass : bool, msg : string} list
    | SomeSelect of {index : int, pass : bool, msg : string} list
    | Input of string
    | SomeInput of string list
  
    val parse : Lexer.token list -> qtype list
  end
\end{Verbatim}

qtype型は,問題または文章を表す型である.Statementは学習者にそのまま表示する文章,
Splitは文章の区切り,Selectは選択問題の各選択肢,SomeSelectは複数選択問題の各選択肢,
Inputは文字列を答える問題の解答,SomeInputは複数の文字列を答える問題の解答を表す.

以下に,sample.txtの字句解析結果を構文解析器Parserで構文解析を行った結果を示す.

\begin{Verbatim}
  # Parser.parse (Lexer.fileLexer "sample.txt");
  val it =
    [
      Statement "問題です.\n",
      Statement "(1) 正解\n",
      Statement "(2) 不正解\n",
      Statement "(3) 不正解\n",
      Statement "\n",
      Select
      [
        {index = 1, msg = "(1)は正解です.\n", pass = true},
        {index = 2, msg = "(2)は不正解です.\n", pass = false},
        {index = 3, msg = "(3)は不正解です.\n", pass = false}
      ]
    ]
    : Parser.qtype list  
\end{Verbatim}

Lexer.StatementはそのままParser.Statementとして変換され,
sample.txt中の,\verb|<select begin>|から,それに対応する\verb|<end>|までの各選択肢
がParser.Selectとして保持されている.

\section{ステージ作成}
\subsection{ステージの作成方法}
DebuQは,事前に作成された準備ファイル群を読み込むことで対話的な学習を実現している.
準備ファイル群は,ステージファイル,ステージリストからなる.
ファイル名がqlistのファイルをステージリスト,ファイル名がq$n$(ただし$n$はステージ番号)
のファイルをステージファイルとする.
付録\ref{app:q1}にステージ1のステージファイルq1を示してある.

ステージリストでは,各ステージのステージ名を定義する.
n行目の文字列をステージ番号nのステージ名とする.
DebuQにおけるステージリストは以下の通りである.

%\VerbatimInput{debuq/qlist}

ステージファイルは,各ステージについて作成する必要がある.
ステージファイルの各行には,StatementかCommandのどちらかを書く.
\begin{itemize}
  \item Command : "\verb|<|"と"\verb|>|"で囲んだ文字列.以下の種類がある.
    \begin{itemize}
    \item \verb|<begin select>|\\
      \verb|<end>|までの各answerについての問題である.\\
      trueの選択肢を答えたら終了する.
    \item \verb|<begin someselect>|\\
      \verb|<end>|までの各answerについての問題である.\\
      選択肢から1つずつ選び,正解を全て答えたら終了する.
    \item \verb|<begin input>|\\
      文字列を入力する問題である.\\
      \verb|<end>|までの文字列と同じものを答えたら終了する.
    \item \verb|<begin someinput>|\\
      空白区切りの複数の文字列を入力する問題である.\\
      正解と同じ入力を行ったら終了する.\\
      順番は合っていなくてもよい.
    \item \verb|<begin answer (int) (bool)>|\\
    問題の選択肢を表す.\verb|<end>|までの文字列を問題回答時のメッセージとする.\\
      \verb|(int)と(bool)|にはそれぞれ数字,真偽値が入る.\\
      例えば\verb|<begin answer 1 false>, <begin answer 2 true>|などとなる.
    \item \verb|<end>|\\
      各begin命令の終了を表す.
    \end{itemize}
  \item Statement : Command以外の文字列
\end{itemize}

以下に,ファイル構成debuq,qlist,q1からなる場合の動作を示す.
ただし,debuqはDebuQの実行ファイル,qlistは文字列"サンプル問題"一行のみからなるファイル,
q1は\ref{sec:implement}におけるsample.txtと内容が同じファイルとする.
\begin{verbatim}
$ ./debuq
1. サンプル問題
タスクを選択してください(q:終了) => 1
問題です.
(1) 正解
(2) 不正解
(3) 不正解

2
(2)は不正解です.
3
(3)は不正解です.
1
(1)は正解です.
Congratulations!
\end{verbatim}

\subsection{本研究で作成されたステージ}
作成したステージと,そのステージをクリアすることで獲得が期待されるデバッグスキルを
表\ref{tbl:stage}に示す.

ステージ1では,単純なバグの例である型エラーを題材にし,デバッグの基礎を学ぶ.
学習者はコンパイルエラーの出力からプログラム中のエラー箇所を推定し,
選択問題によって修正を行う.このステージでは,学習者に学習内容を正確に理解させることより,
問題に取り組む姿勢を持たせることを優先させている.また,間違った選択肢を選んだとしても,
ゲーミフィケーション技術の即時フィードバックに則り,学習者が選んだ選択肢が間違いである理由
とどのように選択肢を選べばよいかを表示する.

ステージ2では,デバッグに限らず多くの問題解決の際に有効な考え方である科学的手法に則った
デバッグの方法を学ぶ.この問題は文献\cite{Andreas:Debugging}の6章「科学的デバッグ」の記述を
参考に作成した.このステージでは,障害が発生した際に仮説を立て,それが正しいかそうでないかを
検証する能力を身につける.

ステージ3では,ステージ2で学習した科学的手法に加えて,変数の依存関係からバグを探し出す方法
について学ぶ.このステージでは,\ref{sec:controlFlow}で述べた制御フローの考え方を,
\SMLSharp ではどのように利用するかを提示する.

ステージ4では,ログの出力を用いてより実践的な仮説の検証を行う方法について学ぶ.
ログの出力については\ref{sec:observation}で述べた通りである.
このステージでは,繰り返し入力を受け付け計算処理を行う再帰関数を定義し,
ログを出力することでプログラム中の欠陥の特定を試みる.

\begin{table}[htp]
  \begin{center}
    \caption{ステージ一覧}
    \label{tbl:stage}
    \begin{tabular}{lrr} \toprule
      ステージ番号 & ステージ名 & デバッグスキル \\ \midrule
      1 & 型エラー & エラー文章の読み取り \\
      2 & 仮説を立てる & 仮説の立て方と検証  \\
      3 & 変数の依存関係 & 各変数の依存関係の確認 \\
      4 & 変数の値の確認 & ログの出力 \\ \bottomrule
    \end{tabular}
  \end{center}
\end{table}

表\ref{tbl:stage}におけるステージは難易度順となっている.
ステージ1が最も簡単で,ステージ4が最も難しい.
デバッグの難易度は一概に決めることは出来ないが,\ref{sec:mechanism}で述べた通り,
バグは欠陥,感染,障害の順に波及していく.
よって,確認された事実がソースコードの欠陥から遠いほどデバッグが難しいと考えられる.
ステージ1では対話環境におけるエラーメッセージから直接欠陥箇所を推定できるが,
ステージ2以降では実行結果誤りを扱っているため,障害から感染を発見し,
感染から欠陥を探す必要がある.
さらに,ステージ2では短いプログラムを先頭から読み進めるが,
ステージ3ではデータ依存,ステージ4では制御依存を考える必要がある.

\chapter{関連研究と議論}
\section{関連研究}
\subsection{学習支援システム}
デバッグ学習を支援する試みとしては,文献\cite{Yama}における授業パッケージと学習支援システムが
存在する.これは,場当たり的なデバッグを行う学習者の学習姿勢を問題であるととらえ,
体系的なデバッグ学習を支援するシステムを提供している.プログラミング言語はJavaを対象とし,
実装誤りの中でも実行結果誤りを引き起こすバグを対象としている.これは,コンパイルエラー
等のエラーメッセージが得られず,バグを特定するのが困難であるためである.また,バグの発見と修正
の2つの作業のうち,バグの発見を学習対象としている.学習する体系的デバッグ手順は,複数の関数
からバグを発見するものであり,プログラムの構造図やデータフローを作成することにより,段階的に
バグを絞り込んでいくものとなっている.

\subsection{シリアスゲーム}
デバッグ学習をテーマにしたシリアスゲームとしては,RoboBUG~\cite{robobug}が存在する.
RoboBUGは,コンピュータサイエンス1年生のデバッグ学習
の導入として開発されたシリアスゲームである.
C++のデバッグを想定しているが,必要に応じて他の言語の
デバッグ学習にも対応できるフレームワークが用意されている.
RoboBUGでは,以下の4つのスキルを習得できる.
\begin{itemize}
  \item Code tracing\\
  コードを読み,適切に動作するかを確認する.
  \item Print statements\\
  プログラム内部に状態を出力するコードを挿入する.
  \item Divide-and-conquer\\
  ソースコードを分割し,バグを独立させる.
  \item Breakpoints\\
  任意のタイミングでプログラムを中断する.
\end{itemize}

また,RoboBUGにはゲーミフィケーション技術におけるアンロックが使用されている.
プレイヤーは条件を満たすと新しいツールを使用することができるようになっている.

\subsection{本研究との相違点}
どちらも,直感的に行われていた従来のデバッグ学習を改善する取り組みであると言える.

文献\cite{Yama}では主に学習支援システムを作成しているが,
ゲーミフィケーション技術を積極的に取り入れているわけではない.
無論,無理に学習をゲーム化する必要はない.
文献\cite{Yama}は,授業パッケージとしての学習支援システムを作成するのであれば,
ゲーミフィケーション技術を使用しなくとも十分な効果が確認された例である.
ゲーミフィケーションは学習者のモチベーションを保つ効果があるが,
学習内容がより専門的になった場合には,
ゲーム要素が学習の妨げになる可能性がある.
例えば,ゲーミフィケーション技術の目標設定は,
具体的な学習内容を提示することで実現が可能である.
しかしこれは学習の自由度が下がることも意味しており,
より柔軟で効率的な学習には従来の非ゲーム的な方法が適当である.

文献\cite{robobug}では主にシリアスゲーム,つまりゲームを作成している.
ゲーム化することのメリットは,
学習者が意欲的にデバッグの学習を行うことができるようになる点である.
しかし,前述したようにゲーム要素を取り入れるほど,
より高度な内容の学習は困難となる.
実際に,シリアスゲームであるRoboBUGはデバッグ学習の導入として開発されており,
その後に改めてデバッグについての学習を行う必要がある.

本研究で開発したDebuQは,これら2つの例の間の立場を取る.
つまり,あくまで開発対象は学習環境システムである点が文献\cite{robobug}におけるRoboBUGとは異なり,
ゲーミフィケーション技術を積極的に取り入れている点で文献\cite{Yama}の研究とは異なる.

\section{議論}
ここでは,DebuQの今後の発展と実用化及び,評価方法に関する考察を行う.

DebuQを導入する利点として,対話的な説明や問題の作成をテキストベースで行うことができる点がある.
しかし,現在は独自の記述方法に従って作成する必要がある.
よって,説明や問題の作成がXMLやJSONといった一般的な形式に対応すれば,
より導入障壁が軽減すると考えられる.

他には,学習者が使用するインターフェイスの改良が挙げられる.
DebuQは,ゲーミフィケーション技術としてアンロックや即時フィードバックを取り入れているが,
その他の技術は採用できていないのが現状である.
よって,他のゲーミフィケーション技術を使用することにより,
より意欲的な学習が期待できる.
DebuQはコンソールアプリケーションとして開発を行ったが,
グラフィック技術に対応することでより多様なゲーミフィケーション技術を取り入れることができると考えられる.
例えば,ステージクリア時のグラフィックによる演出を追加することが,
ゲーミフィケーション技術の称賛演出を取り入れる方法として最も単純で効果的である.
また,グラフィックによる操作の方が直感的であり,
CLIよりもGUIを使用する機会が多い学習者にとっては親しみやすいという利点もある.

内容に関しては,現状のDebuQはデバッグ学習の導入という側面が強いという課題が存在する.
実践的なデバッグ能力を獲得するためには,
実際に自分の手によりログの出力やステートメントの依存関係を追跡することが効果的であると考えられる.
したがって,現状のDebuQの問題の後に,より実践的な課題に取り組む機会を設けることが望ましいと考えられる.
この要素の実現に関する技術的な問題としては,
学習者が作成したプログラムを解釈,実行する機構が必要となることである.
そのためには,例えば既存のコンパイラとの連携を実現する必要があると考えられる.

本研究では研究期間の都合上,評価実験を行うことができなかった.
評価実験を行う場合は,文献\cite{Yama}のように,
プログラミングの基本的な文法を覚えた段階の学習者にDebuQによるデバッグの学習を行わせ,
デバッグ学習の実施前と実施後での試験の成績を確認することで学習効果の評価が行えると考えられる.
試験の内容は,バグが存在するプログラムの修正を行わせるものが考えられる.
DebuQ内で出題される問題と同様のものが考えられるが,
問題に取り組む前の説明文章及び不正解時のコメントを示さない形式で行う.
DebuQによる学習前よりも学習後の方が全体の成績が向上した場合は,
DebuQがデバッグ学習に効果的であると考えられる.

\chapter{結論}
デバッグ学習の促進への第一歩として,
ゲーミフィケーション要素を取り入れたデバッグ学習環境DebuQを開発した.
DebuQはゲーミフィケーション技術「即時フィードバック」を実現するため対話的な学習を実現し,
ゲーミフィケーション技術「アンロック」を用いた段階的な学習順序設計となっている.
学習者は,DebuQを用いることで,
デバッグ技術として「エラー出力の読み取り」「科学的手法」「ステートメントの依存関係」
の3つを学習できる.

しかし,採用できたゲーミフィケーション技術は主にこの2つのみであり,
実用化に向けてはユーザーインターフェースの改良,
学習者が自分の手でログ出力等の作業体験ができる環境への対応が必要である.

\chapter*{謝辞}
本論文の作成にあたり,指導してくださった大堀淳教授,上野雄大准教授に感謝申し上げます.
また,研究方針の決定や本論文の推敲の際に数多くの的確な助言をいただきました
菊池健太郎助教に深く感謝申し上げます.
並びに,研究生活を支えてくださった大堀・上野研究室の皆様に感謝いたします.

\appendix
\chapter{DebuQの実装}
\section{main.sml}
\label{app:main.sml}
%\VerbatimInput{debuq/main.sml}

\section{stage.sml}
\label{app:stage.sml}
%\VerbatimInput{debuq/stage.sml}

\chapter{ステージ1のステージファイルq1}
\label{app:q1}
%\VerbatimInput{debuq/q1}

\bibliographystyle{junsrt}
\bibliography{thesisrefs}

\end{document}

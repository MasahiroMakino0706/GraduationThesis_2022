%#! platex thesisRev; dvipdfmx thesisRev

\documentclass{jreport}
\usepackage{thesis}
% \usepackage{ulem}

% \usepackage{enumerate}
% \usepackage{amsmath}
% \usepackage{url}
% \usepackage{comment}
% \usepackage{booktabs}

% \usepackage[dvipdfmx]{graphicx}
% \usepackage{fancyvrb}
% \fvset{frame=single,numbers=left,numbersep=3pt}

\newcommand\SMLSharp{$\mbox{SML\fontfamily{ptm}\selectfont\#}$}

\title{意味主導の日本語構文解析手法の確立を目指した基礎研究}

\author{牧野雅紘}{B8TB2211}
\eauthor{Masahiro Makino}
\school{東北大学工学部}
\adviser{大堀 淳\ 教授,菊池 健太郎\ 助教}

\date{令和4年1月31日}{令和4年1月}

\renewcommand{\bibname}{参考文献}

\begin{document}
\maketitle


\chapter*{概要}
本研究の目的は, 三上章により体系化された文法論を用いて意味主導で日本語構文を解析することである.
本論文ではまず, 三上章が提唱した文法論である三上文法と日本語の特徴について述べる. 
その後SMLSharpを用いた解析手法を提案し, 具体的なシステム構築の戦略を述べる.
最後にシステムの評価と今後の課題について考察を行う.

\setcounter{tocdepth}{1}
\tableofcontents

\chapter{序論}
\section{背景と目的}
日本語には語順の曖昧さや, 主語が存在しない等の特徴がある. これらの特徴は日本語特有のものであり, 
英語を中心とする西洋の言語にはない特徴である.  そのため西洋言語の構文を高い精度で解析することに長けた
文法主導解析では日本語の構文を十分に解析することは不可能であると考える. 

そこで本研究では, 意味主導で日本語構文を解析することを目的とする.
そのために, 日本語の言語学者である三上章が提唱した三上文法と関数型言語であるSMLSharpを利用し, システム構築を行う.

\section{関連研究}
言語学者である三上章が体系化した文法規則を記した\cite{bunken1}と
東北大学教授, 大堀淳の研究提案草稿\cite{bunken2}を参考文献として使用している


\section{本論文の構成}
本論文の構成は次のとおりである.
2章では本研究で利用する三上文法について紹介する.
3章では三上が紹介する日本語の特徴について述べる.
4章ではSMLSharpにおける自然演算結合であるjoinについて述べ, 意味主導解析の方法について論じる.
5章では具体的なシステム構築の戦略について述べる.
6章ではシステムの評価を考察し6章で今後の課題について述べる.

\chapter{三上文法について}
三上章は日本語で最も重要な文法的手段を助詞の"は"であると述べている.
"は"には2つの役割が存在する. 1つ目は主格ではなく提題, つまりこれから話す内容の題目を提示することであり
2つ目は助詞である"がのにを"の役割を代行することである.
この章では上記に述べた三上文法における助詞の"は"が果たす役割を紹介していく.

\section{提題を表す助詞"は"}
多くの日本語文法において, 助詞の"は"は助詞の"が"と並び主格であるとされている中, 
三上章は, 主格の助詞を"が"のみであると述べている. "は"は提題の役割, つまりこれから話す内容の題目を提示しているのであって
主格ではないと言う. 例を用いて説明しよう.
\begin{quote}
 \begin{itemize}
  \item 象は鼻が長い
  \item わたしは富山の魚が好き
 \end{itemize}
\end{quote}
上の例では, "は"により象という話題について話すことを明示している. 
その後, 鼻に"が"が付帯することで長い主体は鼻であることが説明されている.
従来の日本語文法では"象は"を総主語, "鼻が"を主語とするなどのように主格が2つあることで主語が2つ存在することを認めていた.
三上章は, "象は"が題目, "鼻が"を動作の仕手とし, 主語の存在を否定している. 
下の例では, "わたしは"が題目, "富山の魚"を仕手としている.


\section{他の助詞を代行する助詞"は"}
助詞の"は"は, 他の助詞"がのにを"の役割を代行する.以下に例を示す.
\begin{quote}
 \begin{itemize}
  \item 父はこの本を買ってくれました
  \item 父がこの本を買ってくれました
 \end{itemize}
\end{quote}
   
"は"が最も代行するのが"が"であり, これが"は"と"が"を同じ主格とする考えの根拠になっている.
ただ"は"が文末まで係るにもかかわらず"が"は直後の語幹までしか係らない.
また"は"は他の助詞を代行するが, "が"は他の助詞を代行する役割はない.
そもそも"は"が"が"以外の助詞を代行する場合は"は"を主格とは言えない.
これらの違いは大きく, "は"と"が"を同じ主格として扱ってはいけない根拠となっている.

\begin{quote}
 \begin{itemize}
  \item 去年は夏休みに沖縄に行った
  \item 去年の夏休みに沖縄に行った
 \end{itemize}
\end{quote}

"XはY"において, YがXの性質や消息を表している場合は"XのY"と変換が可能である.

\begin{quote}
 \begin{itemize}
  \item 秋は色々な行事が続く
  \item 秋に色々な行事が続く
 \end{itemize}
\end{quote}
   
"Xは"においてXが時や人, 所の位置を表している場合は"Xの"と変換が可能である.

\begin{quote}
 \begin{itemize}
  \item メバルは煮つけにする
  \item メバルを煮つけにする
 \end{itemize}
\end{quote}

"は"が"を"の役割を代行する際に注意しなければいけないのが, 
"は"は決して動作の仕手を表してはいないということである.例の仕手は料理を実際に行っている人であり, 
決してメバルではない. この例からも"は"を必ずしも主格としてはならないという主張の妥当性が分かるだろう.

\chapter{日本語の特徴}
日本語には主語がない, 連用修飾語の語順は自由である等の特徴があると三上章は述べている.
この章では以上で述べた日本語の特徴について紹介する.
\section{主語のない日本語}
西洋語から輸入された主語は2つの特徴を持つ.

\begin{itemize}
 \item 述語と呼応する
 \item 述語に対する動作主となる
\end{itemize}

例を用いて説明しよう.

\begin{quote}
 \begin{itemize}
  \item She plays baseball.
  \item I play baseball.
 \end{itemize}
\end{quote}

以上のように主語と述語は呼応するため, SheからIに主語が変われば述語の形も変化する.
またplay(plays)の動作主はI(She)であることも明らかである.
日本語における主語と呼ばれているものには, 以上のような特徴を持っていないことを例を使って説明する.

\begin{quote}
 \begin{itemize}
  \item 私は紹介する
  \item 私が紹介する
  \item 私に紹介する
  \item 私を紹介する
 \end{itemize}
\end{quote}

いわゆる主語が変わったとしても述語は変化しない. は格や, が格を違う格で置き換えたとしても述語の変化はない.
つまりは格やが格が述語と呼応していないことが分かる.

\begin{quote}
    象は鼻が長い
\end{quote}

またこの文の述語である長いの動作主は鼻であり, 象ではない.つまりは格は動作主にならない場合がある.
以上のことからは格, が格が"述語と呼応する, 述語の動作主である"の2つの特徴を備えていないことが分かる.
このことから日本語には主語がないと主張が可能である.

\section{連用修飾語の語順は自由}
日本語では連用修飾語の語順が自由である.例をあげて説明しよう.

\begin{quote}
    私はあなたに数学を教える
\end{quote}

(私は)と(あなたに)と(数学を)の3つの連用修飾語は入れ替え可能である.例えば
私は数学をあなたに教えるなどでも日本語文法では可能なのである. これは語順に厳格な英語にはない特徴である.

\chapter{意味主導解析の方法}
意味主導解析において三上文法の他に重要となるものがSMLSharpにおける自然演算結合のjoinである.
この章ではjoinの概略を紹介する.

\section{自然演算結合とは}


\section{}




\chapter{システム構築の戦略}
システムの行う流れは, まず解析したい語において形態素解析を行う. 次に三上文法を使って可能な意味表現を作り出す. 
その後, joinを用いて自然演算結合を行う.
この章では上記の流れを実行するシステム構築の戦略について述べる.
 
\section{juman++による形態素解析}
\subsection{juman++とは}
京都大学院の黒橋研究室により開発がされた形態素解析システムである.言語モデルとしてRecurrent Neural Network Language Model
を利用しているのが特徴的である.
\subsection{形態素解析}
形態素解析により, 単語間の区切りと品詞が分かる. 形態素ごとに以下のような情報を保持したラベル付きレコードを作り, 
そのレコードを要素としたリストを作成する.

\begin{Verbatim}
type jumanOutputTy = 
  {
    表層形 : string, 
    読み : string,
    見出し語 : string,
    品詞大分類 : string,
    品詞大分類_ID : int option,
    品詞細分類 : string,
    品詞細分類_ID : int option, 
    活用型 : string option,
    活用型_ID : int option,
    活用形 : string option,
    活用形_ID : int option,
    意味情報 : jumanOutputSemTy option
  }
\end{Verbatim}

"象は鼻が長い"の形態素解析の結果を以下に示す.

\begin{Verbatim}
    [
        {
          1_表層形 = "象",
          2_読み = "ぞう",
          3_見出し語 = "象",
          4_品詞大分類 = "名詞",
          5_品詞大分類_ID = SOME 6,
          6_品詞細分類 = "普通名詞",
          7_品詞細分類_ID = SOME 1,
          8_活用型 = NONE,
          9_活用型_ID = NONE,
          10_活用形 = NONE,
          11_活用形_ID = NONE,
          12_意味情報 =
            SOME
              {
                カテゴリ = NONE,
                代表表記 = SOME "象/ぞうカテゴリ:動物漢字読み:音",
                反義 = NONE,
                漢字読み = NONE
              }
        },
        {
          1_表層形 = "は",
          2_読み = "は",
          3_見出し語 = "は",
          4_品詞大分類 = "助詞",
          5_品詞大分類_ID = SOME 9,
          6_品詞細分類 = "副助詞",
          7_品詞細分類_ID = SOME 2,
          8_活用型 = NONE,
          9_活用型_ID = NONE,
          10_活用形 = NONE,
          11_活用形_ID = NONE,
          12_意味情報 = NONE
        },
        {
          1_表層形 = "鼻",
          2_読み = "はな",
          3_見出し語 = "鼻",
          4_品詞大分類 = "名詞",
          5_品詞大分類_ID = SOME 6,
          6_品詞細分類 = "普通名詞",
          7_品詞細分類_ID = SOME 1,
          8_活用型 = NONE,
          9_活用型_ID = NONE,
          10_活用形 = NONE,
          11_活用形_ID = NONE,
          12_意味情報 =
            SOME
              {
                カテゴリ = NONE,
                代表表記 =
                  SOME "鼻/はなカテゴリ:動物-部位漢字読み:訓",
                反義 = NONE,
                漢字読み = NONE
              }
        },
        {
          1_表層形 = "が",
          2_読み = "が",
          3_見出し語 = "が",
          4_品詞大分類 = "助詞",
          5_品詞大分類_ID = SOME 9,
          6_品詞細分類 = "格助詞",
          7_品詞細分類_ID = SOME 1,
          8_活用型 = NONE,
          9_活用型_ID = NONE,
          10_活用形 = NONE,
          11_活用形_ID = NONE,
          12_意味情報 = NONE
        },
        {
          1_表層形 = "長い",
          2_読み = "ながい",
          3_見出し語 = "長い",
          4_品詞大分類 = "形容詞",
          5_品詞大分類_ID = SOME 3,
          6_品詞細分類 = "*",
          7_品詞細分類_ID = NONE,
          8_活用型 = SOME "イ形容詞アウオ段",
          9_活用型_ID = SOME 18,
          10_活用形 = SOME "基本形",
          11_活用形_ID = SOME 2,
          12_意味情報 =
            SOME
              {
                カテゴリ = NONE,
                代表表記 =
                  SOME "長い/ながい反義:形容詞:短い/みじかい",
                反義 = NONE,
                漢字読み = NONE
              }
        }
      ]
\end{Verbatim}

Juman++の形態素解析結果が未定義の場合はNONEとしている.

\section{文節(名詞+助詞)に分ける}
文節に分けることで, 名詞に対して助詞の果たす格の候補を特定できる.
"が"であればが格つまり主格の役割をはたしている.
"は"であれば"は"がのにを"を代行する事実から, が格, の格, に格またはを格が候補となる.
このように考えられる格の種類によって表現されうる文の意味表現を列挙することが可能である.

形態素解析の結果, 各形態素の品詞は特定できているため, 形態素の集合を文節の集合に分けることは容易である.
以下のような関数を定義して文節分けを実現した.

\begin{Verbatim}
    fun separate record lis = 
    if lis = nil then
      case #4_品詞大分類 record of
        "形容詞" => [Type.形容詞 {形容詞 = #1_表層形 record, 形容詞情報 = record}]
      | "動詞" => [Type.動詞 {動詞 = #1_表層形 record, 動詞情報 = record}]
      | "名詞" => [Type.名詞 {名詞 = #1_表層形 record, 名詞情報 = record}]
      | _ => raise Fail "unknown hinshi"
    else
      if #4_品詞大分類 (List.hd lis) = "助詞" then
        case #1_表層形 (List.hd lis) of
          "は" => List.@ ([Type.名詞は {名詞 = #1_表層形 record, 名詞情報 = record, 助詞情報 = List.hd lis}], separate (List.hd (List.tl lis)) (List.tl (List.tl lis)))
        | "が" => List.@ ([Type.名詞が {名詞 = #1_表層形 record, 名詞情報 = record, 助詞情報 = List.hd lis}], separate (List.hd (List.tl lis)) (List.tl (List.tl lis)))
        | "の" => List.@ ([Type.名詞の {名詞 = #1_表層形 record, 名詞情報 = record, 助詞情報 = List.hd lis}], separate (List.hd (List.tl lis)) (List.tl (List.tl lis)))
        | "に" => List.@ ([Type.名詞に {名詞 = #1_表層形 record, 名詞情報 = record, 助詞情報 = List.hd lis}], separate (List.hd (List.tl lis)) (List.tl (List.tl lis)))
        | "を" => List.@ ([Type.名詞を {名詞 = #1_表層形 record, 名詞情報 = record, 助詞情報 = List.hd lis}], separate (List.hd (List.tl lis)) (List.tl (List.tl lis)))
        | "へ" => List.@ ([Type.名詞へ {名詞 = #1_表層形 record, 名詞情報 = record, 助詞情報 = List.hd lis}], separate (List.hd (List.tl lis)) (List.tl (List.tl lis)))
        | _ => raise Fail "unknown hinshi"
      else
        case #4_品詞大分類 record of
          "形容詞" => List.@ ([Type.形容詞 {形容詞 = #1_表層形 record, 形容詞情報 = record}], separate (List.hd lis) (List.tl lis))
        | "動詞" => List.@ ([Type.動詞 {動詞 = #1_表層形 record, 動詞情報 = record}], separate (List.hd lis) (List.tl lis))
        | "名詞" => List.@ ([Type.名詞 {名詞 = #1_表層形 record, 名詞情報 = record}], separate (List.hd lis) (List.tl lis))
        | _ => raise Fail "unknown hinshi"
\end{Verbatim}

形態素の品詞をレコードから取り出し, 助詞であれば前に存在する名詞と助詞の情報を統合し, 文節に分けることを実現している.

以下に"象は鼻が長い"の文節分け結果を示す.

\begin{Verbatim}
    [
        名詞は
        {
          助詞情報 =
            {
              1_表層形 = "は",
              2_読み = "は",
              3_見出し語 = "は",
              4_品詞大分類 = "助詞",
              5_品詞大分類_ID = SOME 9,
              6_品詞細分類 = "副助詞",
              7_品詞細分類_ID = SOME 2,
              8_活用型 = NONE,
              9_活用型_ID = NONE,
              10_活用形 = NONE,
              11_活用形_ID = NONE,
              12_意味情報 = NONE
            },
          名詞 = "象",
          名詞情報 =
            {
              1_表層形 = "象",
              2_読み = "ぞう",
              3_見出し語 = "象",
              4_品詞大分類 = "名詞",
              5_品詞大分類_ID = SOME 6,
              6_品詞細分類 = "普通名詞",
              7_品詞細分類_ID = SOME 1,
              8_活用型 = NONE,
              9_活用型_ID = NONE,
              10_活用形 = NONE,
              11_活用形_ID = NONE,
              12_意味情報 =
                SOME
                  {
                    カテゴリ = NONE,
                    代表表記 =
                      SOME "象/ぞうカテゴリ:動物漢字読み:音",
                    反義 = NONE,
                    漢字読み = NONE
                  }
            }
        },
        名詞が
        {
          助詞情報 =
            {
              1_表層形 = "が",
              2_読み = "が",
              3_見出し語 = "が",
              4_品詞大分類 = "助詞",
              5_品詞大分類_ID = SOME 9,
              6_品詞細分類 = "格助詞",
              7_品詞細分類_ID = SOME 1,
              8_活用型 = NONE,
              9_活用型_ID = NONE,
              10_活用形 = NONE,
              11_活用形_ID = NONE,
              12_意味情報 = NONE
            },
          名詞 = "鼻",
          名詞情報 =
            {
              1_表層形 = "鼻",
              2_読み = "はな",
              3_見出し語 = "鼻",
              4_品詞大分類 = "名詞",
              5_品詞大分類_ID = SOME 6,
              6_品詞細分類 = "普通名詞",
              7_品詞細分類_ID = SOME 1,
              8_活用型 = NONE,
              9_活用型_ID = NONE,
              10_活用形 = NONE,
              11_活用形_ID = NONE,
              12_意味情報 =
                SOME
                  {
                    カテゴリ = NONE,
                    代表表記 =
                      SOME "鼻/はなカテゴリ:動物-部位漢字読み:訓",
                    反義 = NONE,
                    漢字読み = NONE
                  }
            }
        },
        形容詞
        {
          形容詞 = "長い",
          形容詞情報 =
            {
              1_表層形 = "長い",
              2_読み = "ながい",
              3_見出し語 = "長い",
              4_品詞大分類 = "形容詞",
              5_品詞大分類_ID = SOME 3,
              6_品詞細分類 = "*",
              7_品詞細分類_ID = NONE,
              8_活用型 = SOME "イ形容詞アウオ段",
              9_活用型_ID = SOME 18,
              10_活用形 = SOME "基本形",
              11_活用形_ID = SOME 2,
              12_意味情報 =
                SOME
                  {
                    カテゴリ = NONE,
                    代表表記 =
                      SOME "長い/ながい反義:形容詞:短い/みじかい",
                    反義 = NONE,
                    漢字読み = NONE
                  }
            }
        }
      ]
\end{Verbatim}


\section{文節ごとに部分意味表現を作成する}
文節ごとの意味を保持した表現(部分意味表現と呼ぶ)を作る. 部分意味表現を作成するにあたって, 
"は"が"がのにを"を代行すること, "は"は文の提題を行うことを利用する. 名詞+はを検出した際に

\section{部分意味表現を自然結合演算}


\chapter{評価}

\chapter{今後の課題}
結論

\chapter*{謝辞}
{謝辞}

\appendix
\chapter{Appendix}
Appendix
\section{sec1}
sec1


% \bibliographystyle{junsrt}
% \bibliography{thesisrefs}

\begin{thebibliography}{10}

\bibitem{bunken1}
三上章.
\newblock 象は鼻が長い.
\newblock くろしお出版, 2021

\bibitem{bunken2}
大堀淳.
\newblock 意味論主導の日本語構文解析について.
\newblock unpublished manuscript, 2022.

\bibitem{robobug:2017}
Jeremy S.~Bradbury Michael A.~Miljanovic.
\newblock Robobug: A serious game for learning debugging techniques.
\newblock In {\em the 2017 ACM Conference}, 2017.

\end{thebibliography}

\end{document}
